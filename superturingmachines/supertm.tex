\section{Superturingmaschinen}

\begin{frame}[c]{Superturingmaschinen: Intro}
    \Large
    Eigentlich eine Normale Turingmaschine. \\
    \pause
    Wir haben als Zeitschritte Ordinalzahlen.
    % Dadurch ergeben sich jetzt erstmal schon so ein paar Fragen,
    % also nach dem Zustand zum Zeitpunkt $\omega$ oder ähnliches.
\end{frame}

\subsection{Grenzverhalten}

\begin{frame}[c]{Grenzverhalten - Unklarheiten}
    \Large
    Offene Fragen sind:
    \begin{itemize}
            \pause
        \item In welchem Zustand sind wir?
            \pause
        \item Wie sieht das Band aus?
            \pause
        \item Was bedeutet das?
    \end{itemize}
\end{frame}


\begin{frame}[c]{Grenzverhalten - Möglichkeiten}
    \Large
    Zwei Möglichkeiten:
    \begin{itemize}
            \pause
        \item Wir halten.
            \pause
            Das ist einfach :)
            \pause
        \item Wir halten nicht.
    \end{itemize}
\end{frame}


\begin{frame}[c]{Grenzverhalten - Erklärung}
    \Large
    \pause
    Es ist echt verdammt schwer GIFs in PDFs zu bekommen ... \\
    \pause
    Demotime.
    % Zeige vier Varianten: Immer 1 -> Klar, 1
    % Immer 0 -> Klar, 0
    % Endlich oft 1, unendlich oft 0: immernoch 0
    % Unendlich oft 1, unendlich oft 0: ist 1
    % \pause
    % { \small
    % (Hier werde ich anhand von externen Bildern erklären, wie das Grenzverhalten für Zellen zu verstehen ist)
    % }
\end{frame}


\begin{frame}[c]{Grenzverhalten - Beispiel}
    \normalsize
%    Turingmaschine:
    \code{Prüfe im Start- und Limeszustand, ob die aktuelle Zelle eine Eins enthält.
    \begin{itemize}
        \item Wenn Ja, dann halte.
        \item Wenn nein, dann lass die Zelle aufleuchten und laufe ohne zu halten nach rechts.
    \end{itemize}}
    \pause
    Scheint sich zu wiederholen, hält aber nach Schritt $\omega^2$.

    \bigskip
    \pause
    Eine Superturingmaschine wiederholt sich genau dann, wenn
    \begin{itemize}
        \item die Aufnahmen zu zwei Limesordinalzahlen gleich sind und
        \item zwischen diesen Zeiten keine Zellen, die Null waren zu eins werden.
    \end{itemize}
\end{frame}


\subsection{Fähigkeiten}

\begin{frame}[c]{Fähigkeiten}
    \large
    \begin{itemize}
        \item Alles was Normale Turingmaschinen können
            \pause
        \item Das Klassische Halteproblem lösen
            \pause
        \item Gewisse Zahlentheorethische Aussagen entscheiden
            \pause
        \item Turingmaschinen mit gewissen Fähigkeiten finden
            \pause
        \item Funktionen mit gewissen Eigenschaften finden
            \pause
        \item Die Klasse der Wohlordnungen entscheiden
    \end{itemize}
\end{frame}


\begin{frame}[c]{Fähigkeiten II}
    \Large
    Was Superturingmaschinen dennoch nicht können:
    \begin{itemize}
        \item Beliebige 0/1-Folgen auf das Band schreiben
            \pause
        \item Ihr eigenes Halteproblem lösen
            \pause
        \item Beliebig komplexe Aussagen entscheiden
            \pause
        \item Kaffe kochen
            \pause
        \item ...
    \end{itemize}
\end{frame}




