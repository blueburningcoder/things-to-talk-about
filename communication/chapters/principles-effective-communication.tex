\section{Prinzipien für Effektive Kommunikation}

\subsection{Verstehe die Situation des anderen}

\begin{frame}[c]{Verstehe die Situation des anderen: Problematik}
    \large
    Problem:
    \begin{itemize}[<+(1)->]
        \item Es wird aneinander vorbei geredet
        \item Versuch, die eigene Position klar zu machen
        \item Kein Versuch, eine gemeinsame Lösung zu finden
        \item Unterschiedliche Entscheidungsgewalt: Unzufriedenheit
        \item Gefühl nicht verstanden zu werden: weitere Ursache
    \end{itemize}
\end{frame}

\begin{frame}[c,standout]
    Was für Erwartungen habt ihr?
\end{frame}

\begin{frame}[c]{Erwartungen: Notizen}
    
\end{frame}

\begin{frame}[c]{Verstehe die Situation des anderen: Lösung}
    \large
    Lösung:
    \begin{itemize}[<+(1)->]
        \item Versuche nicht, die eigene Position klar zu machen
        \item Versuche, die andere Position vollständig zu verstehen
        \item Versuche, einen Weg zu finden, mit dem alle Beteiligten zufrieden sind
        \item Aktives Zuhören
    \end{itemize}
\end{frame}


\begin{frame}[c]{Technik: Aktives Zuhören}
    \large
    Hinweise:
    \begin{itemize}[<+(1)->]
        \item Konzentriere dich vollständig darauf, was dein Gegenüber sagt
        \item Interpretiere, was gesagt wurde
        \item Beachte nonverbale Hinweise
        \item Werte nicht (don't judge)!
        \item Frage nach Erläuterung
        \item Sei geduldig, unterbrich nicht
    \end{itemize}
\end{frame}


\subsection{Gemeinsame Nenner finden}
\begin{frame}[c]{Gemeinsamer Nenner finden: Problematik}
    \large
    Problem:
    \begin{itemize}[<+(1)->]
        \item Personen haben manchmal Ziele außerhalb
        \item Ein Wichtigkeits-Ego bei Verantwortlichen
        \item Ego/Machtkämpfe untereinander
        \item Scheinbar gegensätzliche Positionen
    \end{itemize}
    \pause
    Relativ eindeutig bei:
    \begin{itemize}[<+(1)->]
        \item Vereinen
        \item Den meisten {\em zielgerichteten} Treffen
    \end{itemize}
\end{frame}

\begin{frame}[c]{Gemeinsamen Nenner finden: Lösung}
    \large
    Vorteile:
    \begin{itemize}[<+(1)->]
        \item Verhindert 'Blame Game'
        \item Verständnis für Ziele und Intentionen
        \item Das gemeinsame Ziel ist meist wichtiger als \\kleine Streitigkeiten oder Ego
    \end{itemize}
    \vspace{1cm}

    \pause
    \begin{block}{Beispiel}
        Wir sind alle hier um Studierenden zu helfen, oder 
        um die Fachschaft zu unterstützen.
    \end{block}
\end{frame}
