\subsection{Nach Feedback fragen}

\begin{frame}[c]{Feedback: Situation}
    \large
    Problem:
    \begin{itemize}[<+(1)->]
        \item Meist werden Gedanken bei sich selbst behalten
        \item Unklarheiten werden als unwichtig abgetan
    \end{itemize}
    \pause
    Vorteile:
    \begin{itemize}[<+(1)->]
        \item Möglichkeit, Wichtiges (Blind spots!) zu lernen
        \item Bessere Reaktion auf und Umgang mit Kritik
        \item Man selbst hat die Verantwortlichkeit für Erfolg
    \end{itemize}
\end{frame}

\begin{frame}[standout]
    Was für Feedback habt ihr an uns?
\end{frame}

\begin{frame}[c]{Feedback: Notizen}
    
\end{frame}

\begin{frame}[c]{Feedback: Lösung}    
    \large
    Lösung:
    \begin{itemize}[<+(1)->]
        \item Feedback einfach akzeptieren
        \item Nicht rechtfertigen, nur verstehen
        \item Frage nach einer Bestätigung von Aufträgen
        \item Frage nach Zusammenfassungen der Interpretation
        \item Frage nach offenen (anonymen) Feedback-Runden
    \end{itemize}
\end{frame}

% Next TODO: Slides zu gut feedback geben, nicht vorwurfsvoll etc + übung


\begin{frame}[standout]
    Pause: 5min
\end{frame}




\section{Interaktiver Teil: selbst Ausprobieren}

\begin{frame}[c]{Übung: Die andere Seite verstehen}
    \large
    \begin{block}{Beschreibung}
    Breakout-Sessions mit jeweils 2 oder 3 Personen und 15min Zeit.
    Das Ziel ist, die Position einer Person zu einem vorgegebenen Thema
    für die befragte Person zufriedenstellend wiedergeben zu können.
    Versucht, Aktiv zuzuhören.
    \end{block}
    Themen: \url{https://demo.hedgedoc.org/xoHcMXPtTryNGooWzjFSGQ} (Vorschlag)
\end{frame}
