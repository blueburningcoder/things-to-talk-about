
\begin{frame}[c]{Wahr 'nach Definition'}
    You try to establish any sort of empirical proposition as being true "by definition".  Socrates is a human, and humans, by definition, are mortal.  So is it a logical truth if we empirically predict that Socrates should keel over if he drinks hemlock?  It seems like there are logically possible, non-self-contradictory worlds where Socrates doesn't keel over - where he's immune to hemlock by a quirk of biochemistry, say.  Logical truths are true in all possible worlds, and so never tell you which possible world you live in - and anything you can establish "by definition" is a logical truth.
\end{frame}



\begin{frame}[c]
    You try to define a word using words, in turn defined with ever-more-abstract words, without being able to point to an example.  "What is red?"  "Red is a color."  "What's a color?"  "It's a property of a thing?"  "What's a thing?  What's a property?"  It never occurs to you to point to a stop sign and an apple.  (Extensions and Intensions.)
\end{frame}


\begin{frame}[c]
    The extension doesn't match the intension.  We aren't consciously aware of our identification of a red light in the sky as "Mars", which will probably happen regardless of your attempt to define "Mars" as "The God of War".  (Extensions and Intensions.)
\end{frame}


\begin{frame}[c]{Außnahmen}
    A verbal definition works well enough in practice to point out the intended cluster of similar things, but you nitpick exceptions. Not every human has ten fingers, or wears clothes, or uses language; but if you look for an empirical cluster of things which share these characteristics, you'll get enough information that the occasional nine-fingered human won't fool you.  (The Cluster Structure of Thingspace.)
\end{frame}


\begin{frame}[c]{Diskutieren über Definitionen}
    You allow an argument to slide into being about definitions, even though it isn't what you originally wanted to argue about. If, before a dispute started about whether a tree falling in a deserted forest makes a "sound", you asked the two soon-to-be arguers whether they thought a "sound" should be defined as "acoustic vibrations" or "auditory experiences", they'd probably tell you to flip a coin.  Only after the argument starts does the definition of a word become politically charged.  (Disputing Definitions.)
\end{frame}


\begin{frame}[c]
    You get into arguments that you could avoid if you just didn't use the word. If Albert and Barry aren't allowed to use the word "sound", then Albert will have to say "A tree falling in a deserted forest generates acoustic vibrations", and Barry will say "A tree falling in a deserted forest generates no auditory experiences".  When a word poses a problem, the simplest solution is to eliminate the word and its synonyms.  (Taboo Your Words.)
\end{frame}


\begin{frame}[c]
    Noch weitere beispiele zu 'aneinander vorbeigeredet':
    \begin{itemize}[<+(1)->]
        \item proposing solutions before fully understanding the problem
        \item 
    \end{itemize}
\end{frame}


