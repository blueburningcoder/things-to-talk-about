\section{Wohlordnungen}


\begin{frame}[c]{Wohlordnungen}
    \Large
    Eine Wohlordnung ist eine total fundierte Ordnungsrelation.
    \newline
    \bigskip
    \pause
    Was heißt das?
\end{frame}


\begin{frame}[c]{Wohlordnung: Beispiel}
    \Large
    $\mathbb{N}: \{0, 1, 2, .. \}$ \only<5->{(Order-Type: $\omega$)}\\
    \pause
    $\mathbb{Z}: \{0, -1, 1, -2, 2,..\}$ \only<6->{(Order-Type: $\omega$)} \\
    \pause
    $\mathbb{Z}: \{0, 1, 2, 3,.., -1, -2, -3,.. \}$ \newline
    \only<7->{(Order-Type: $2 * \omega$)}
    \newline
    \newline
    \pause
    \{
    \normalsize
    \begin{tabular}{|llllll}
        \hline
        0 & 1 & 3 & 6 & 10 & ..\\
        2 & 4 & 7 & 11 & 16 & .. \\
        5 & 8 & 12 & 17 & 23 & .. \\
        9 & 13 & 18 & 24 & .. & .. \\
        .. & .. & .. & .. & .. & ..
    \end{tabular}
    % $\mathbb{R}: \{2^{-n} - 2^{-m -n}\ |\ 0 \leq m,n \in \mathbb{N} \}$ \newline
    \Large
    \}
    \only<8->{(Order-Type: $\omega^2$)}
\end{frame}

