\usepackage{etex}
\usepackage{graphicx}
\usepackage[export]{adjustbox}
\usepackage{multicol}
\usepackage{pdfpcnotes}
\usepackage{pdfpages}
% \usepackage[dvipsnames]{xcolor}


% \usepackage{minted}
% \usemintedstyle{pastie}


\usetheme[numbering=counter, progressbar=frametitle, subsectionpage=progressbar]{metropolis}


\date{December 29, 2020}
% \date{12. Juli 2019}


% \institute{Add Your Institute here}
% \titlegraphic{\vspace{4cm} \hspace{7cm} \includegraphics[height=2cm]{Logo_INST}}
% \titlegraphic{\vspace{4cm} \hspace{7cm} \Huge\LaTeX}

\makeatletter
\patchcmd{\beamer@sectionintoc}
  {\vfill}
  {\vskip0\itemsep}
  {}
  {}
\pretocmd{\beamer@sectionintoc}   {\vskip0\itemsep}{}{}
\patchcmd{\beamer@sectionintoc}{\vskip1.5em}{\vskip0.5em}{}{} % <- this one finally works.
\makeatother


\iftwocols
\AtBeginSection[]
{
\begin{frame}[c, plain]
  \begin{minipage}{22em}
    \usebeamercolor[fg]{section title}
    \usebeamerfont{section title}
    % \insertsectionhead\\[-1ex]
    \begin{multicols}{2}
    \tableofcontents[currentsection, hideallsubsections]
    \end{multicols}
    % \usebeamertemplate*{progress bar in section page}
  \end{minipage}
\end{frame}
}
\else
\AtBeginSection[]
{
\begin{frame}[c, plain]
  \begin{minipage}{22em}
    \usebeamercolor[fg]{section title}
    \usebeamerfont{section title}
    % \insertsectionhead\\[-1ex]
    \tableofcontents[currentsection, hideallsubsections]
    % \usebeamertemplate*{progress bar in section page}
  \end{minipage}
\end{frame}
}

\fi

\AtBeginSubsection[]
{
\begin{frame}[c,plain]
  \begin{minipage}{22em}
    \raggedright
    \usebeamercolor[fg]{section title}
    \usebeamerfont{section title}
    \insertsectionhead\\[-1ex]
    % \begin{multicols}{2}
    % \tableofcontents[sectionstyle=show/hide,subsectionstyle=hide/hide/hide]
    % \end{multicols}

    \usebeamertemplate*{progress bar in section page}
    \par
    \ifx\insertsubsectionhead\@empty\else%
      \usebeamercolor[fg]{subsection title}%
      \usebeamerfont{subsection title}%
      \vspace{0.5cm}
      \tableofcontents[sectionstyle=hide/hide,subsectionstyle=show/shaded/hide]
    \fi
  \end{minipage}
\end{frame}
}

\begin{document}

\maketitle

\addtocounter{framenumber}{1}

% multicols from:
% https://tex.stackexchange.com/questions/24343/splitting-toc-into-two-columns-on-single-frame-in-beamer

%%%%%%%%%%%%%%%%%%%%%%%%%%%%%%%%%%%%%%%%%%%%%%%%%%%%%%%%%%%%%%%%%%%%%%%%%%%%%%%%%%%%%%%%%%%%%%%%%%%%%%%%%%%%%%%%%%%

\iftwocols

\begin{frame}[c,plain]% {Agenda}
    \begin{minipage}{22em}
    \usebeamerfont{section title}
    \usebeamercolor[fg]{section title}
    \begin{multicols}{2}
    \tableofcontents[hideallsubsections]
%   \tableofcontents[]
    \end{multicols}
    \end{minipage}
    % \clearpage
\end{frame}

\else

\begin{frame}[c,plain]% {Agenda}
    \begin{minipage}{22em}
    \usebeamerfont{section title}
    \usebeamercolor[fg]{section title}
    \tableofcontents[hideallsubsections]
%   \tableofcontents[]
    \end{minipage}
    % \clearpage
\end{frame}
\fi


\newcommand{\code}[1]{
    \begin{center}
    \setlength{\fboxrule}{1pt}
    \setlength{\fboxsep}{8pt}
        {\fbox{\parbox{0.81\textwidth}{#1}}}
   \end{center}
}

\newenvironment{codeboxed}[1]
        {\begin{minipage}{\linewidth}\begin{center}#1\\[1ex]\begin{tabular}{|p{\textwidth}|}\hline}
        {\\\hline\end{tabular}\end{center}\end{minipage}}


\newcommand{\backupbegin}{
   \newcounter{finalframe}
   \setcounter{finalframe}{\value{framenumber}}
}

\newcommand{\backupend}{
   \setcounter{framenumber}{\value{finalframe}}
}


\newcommand{\mailto}[1]{
    \href{mailto:#1}{#1}
}

\newcommand{\todo}[1]{
    {\Large\color{red}{(TODO: #1)}}
}

% \definecolor{green1}{RGB}{38, 69, 37} % #264525
\definecolor{green1}{RGB}{72, 129, 69} % #488145
% \definecolor{blue1}{RGB}{7, 43, 94} % #072b5e
\definecolor{blue1}{RGB}{14, 82, 179} % #0e52b3
% \definecolor{violet1}{RGB}{58, 38, 68} % #3a2644
\definecolor{violet1}{RGB}{108, 72, 126} % #6c487e
% \definecolor{orang1}{RGB}{, , 0} % #663400
\definecolor{orang1}{RGB}{193, 98, 0} % #c16200


\newcommand{\green}[1]{
    \textcolor{green1}{#1}
}
\newcommand{\blue}[1]{
    \textcolor{blue1}{#1}
}
\newcommand{\vio}[1]{
    \textcolor{violet1}{#1}
}
\newcommand{\orang}[1]{
    \textcolor{orang1}{#1}
}
