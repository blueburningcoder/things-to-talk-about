



\subsection{Assumed Root Causes}

\begin{frame}[c]{Problem: Many Theories}
    \large
    \begin{itemize}[<+(1)->]
        \item Everything is interlinked
        \item Very hard to distinguish cause and effect
        \item At least one Theory for every Hallmark
        \item Every prestigious lab has its own Theory
        \item A lot of speculation on all sides
        \item Unclear if we can already see the full picture
    \end{itemize}
\end{frame}


\addtocounter{framenumber}{1}
\begin{frame}[standout]
    Disclaimer: Purely Speculation including many Unknowns
\end{frame}

\begin{frame}[c]{Mitochondrial dysfunction}
    \large
    Turns out, mitochondrial dysfunction accounts for telomere-dependent senescence \cite{passos2007mitochondrial}.
\end{frame}


\begin{frame}[c]
    \large
    Assumed root causes: free radicals and transposon damage \\
    Maybe not in too much detail? Could fill 30min itself \cite{CorePath13:online} \\
    \pause

    p21 and reactive oxygen feedback for senescence \cite{passos2010feedback}
\end{frame}



\subsection{Open Questions}

\begin{frame}[c]{Questions Unanswered}
    \begin{itemize}[<+(1)->]
        \item Where are the ROS produced? Mitochondria are the top candidate - there’s a known mechanism for ROS production by mitochondria, as well as experimental evidence that mitochondrion-targeted antioxidants specifically reduce ROS-induced damage.
        \item How do the ROS and/or damaged molecules move between compartments, e.g. nucleus/cytoplasm/extracellular? I have seen very little on this, and consider it a major blindspot. I’m not sure if it’s a blindspot for the field or if I just haven’t found the right cluster of papers.
        \item Are the quantitative changes in DNA/protein/fat damage compatible with a single underlying cause? Do they match plausible estimates of ROS from dysfunctional mitochondria? Again, I haven’t seen Fermi estimates here, but I’d like to.
        \item Why is the immune system degradation related? It does degrade similarly over time, does it 'age' as well? This implies at least a second independent pathway and approach
    \end{itemize}
\end{frame}
