\section{What can I do?}

\subsection{Pharmacological}

\addtocounter{framenumber}{1}
\begin{frame}[standout]
    \LARGE
    This is NOT Medical Advice!
\end{frame}

\begin{frame}[c]{Pharmacological}
    List of medications taken regularly by anti-aging researchers:

    \begin{itemize}[<+(1)->]
        \item Metformin - calorie restriction mimetic that controls blood sugar
            % ist in deutschland diabetes typ2 medikament: erste wahl
            % actually helpful for type 1 as well, but new knowledge
            % vorteil: hohe senkung von Hba1c-Wert (prozentual) (sollte bei gesund unter 6%, niedriger ist besser)
            % -> verschreibungspflichtig
            % nachteil: wirkmechamismus nicht vollständig bekannt
            %   hemmt glykosebildung (aus pyruvat) in leber
            %   seltene nebenwirkung: lactatacetose (lactat wird nicht zu zucker umgebaut), zu 50% tödlich
            % sehr potent, aber unklar wie viel es beim Menschen hilft
            % kritisch für Umwelt weil kann nicht abgebaut werden (verbrennen!)
            % hormonähnliche Wirkung, aber günstig: weniger als 1EUR/TAG
            % alternative: glieflocine (?) mit mehr vorteilen, aber viel teurer
        \item Quercetin - anti-aging flavenoid that acts as a senolytic
            % flavenoid: sekundäre pflanzenstoffe (z.B. Frost-, oder Lichtschutz)
            % hier: in Roter traube/Rotwein (teil Farbstoff) aber auch andere
            % 'soll antioxidativ wirken' (umstritten, insb. relevanz)
            % aber: wenig kommt an mit sehr kurzer wirkung, schwer wirkung nachzuweisen
            % verschreibungsfrei
        \item Resveratrol - sirtuin enzyme activator and calorie restriction mimetic
            % ähnlich einem flavonoid
            % verschreibungsfrei
        \item Vitamin D - blood tested to optimize, ideally 2000IU per day
            % meisten haben mangel, schadet nicht aufzufüllen
            % tatsächlich ein weitreichendes Hormon
            % 60-80μg/ml im Serum ideal, alles andere: erhöhtes sterblichkeitsrisiko
            % fettlösliches 'Vitamin', aber überdosierung enorm schwierig
            % bekommt man überall, billig
        \item Vitamin B12 - as many people are deficient
            % auch weit verbreiteter Mangel, organische substanz
            % tägl. bedarf: 2-3μg
            % insb. bei Fleischarmer Ernährung wichtig zu supplementieren
            % nur kleiner teil wird auch verwertet
            % bekommt man überall, billig
    \end{itemize}
\end{frame}


\begin{frame}[c]{Pharmacological II}
    \large
    On the more extreme end (for older people or people with a higher risk tolerance):

    \begin{itemize}[<+(1)->]
        \item Rapamycin - an mTOR inhibitor that attenuates senescence
            % immunsuppresiv
            % mTOR ist mega wichtig, hemmt proliferation aber auch immunabwehr (?)
            % wird nicht mehr eingesetzt
            % alternative: imatinib, auch mTOR-pfad-relatiert
            % auch nicht über verschreibungen, sehr teuer, gibt alternativen (auch teuer)
        \item NAD-boosters such as NMN (Nicotinamide) and NR - enhancers of stem cell function
            % NAD ist cofactor in versch. zeug (auch: Vit B2, D), hier: B3
            % bilden redox-cofactoren, zum elektronentransport, erlaubt sehr spezielle reaktionen
            % sirtruine konsumieren NAD, unklar wie viel das bringt
            % verschreibungsfrei, sehr günstig: Vit-B-Komplex (B2/B3) (alt: nur B3)

            % spermidin - soll telomerabbau verlangsamen (in vitro!)
            %   ist schon in gameten, unklar ob mehr hilft
            %   verschreibungsfrei, aber nicht sehr günstig
        \item Dasatinib - a senolytic usually used in combination with quercetin
            % auch mTOR-inhibitor
            % verschreibungspflichtig
    \end{itemize}
    % alles minimale effekte, viel wichtiger:
    % - gesunde & ausgewogene ernährung
    % - sport
    % - wenig stress
    % - keine depri
    \pause
    \LARGE
    \textbf{But: a balanced lifestyle will get you much further}
\end{frame}


\subsection{Lifestyle}

\begin{frame}[c]{Lifestyle is more important}
    \large
    Available medication can add only so much, much more important are:

    \begin{itemize}[<+(1)->]
        \item Healthy and balanced diet \cite{willcox2007caloric}
        \item Regular Exercise \cite{lee1995exercise}
        \item Low-Stress Environment
        \item Close friends \cite{olsen1991social}
        \item Fulfilling Life \cite{diener2011happy}
        \item Not suffering from depression \cite{cuijpers2002excess}
    \end{itemize}
    \pause
    The statistical evidence is clear on this!
\end{frame}
