\section{Die Gleichheit der Menschen}


\begin{frame}[c]{Was ist Gleichheit zwischen Menschen?}
    \Large
    Mehr als nur Gleichheit zwischen Äpfeln. \\ \pause
    Weil Menschen so viel mehr Eigenschaften haben.
\end{frame}


\begin{frame}[c]{Sind Menschen wirklich Gleich?}
    \Large

    Nein sind wir nicht.

    \begin{itemize}
            \pause
        \item Genderfeindlich
            \pause
        \item Ignorant
            \pause
        \item Arrogant
            \pause
        \item Inkonsistent
    \end{itemize}

\end{frame}


\begin{frame}[c]{Zitat von Fukuzawa Yukichi}
    \Large
    It is said that heaven does not create one man above or below another man.
    Any existing distinction between the wise and the stupid, between the rich
    and the poor, comes down to a matter of education.
\end{frame}


\begin{frame}[c]{Gleichheit zwischen Menschen}
    \Large
    Versteht mich nicht falsch, das heißt immernoch dass alle Menschen (z.B. vor
    Gericht) gleich Behandelt werden sollen, also den Umständen entsprechend.
\end{frame}


\begin{frame}[c]{Utilitarismus}
    \Large
    Idee: Es soll allen Menschen möglichst gut gehen.
\end{frame}








