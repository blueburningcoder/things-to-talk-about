\section{Overview}


\begin{frame}[c]{What is HTM?}
    \Large
    \begin{itemize}[<+(1)->]
        \item biologically constrained \textbf{theory of intelligence}
        \item originally described in ''On Intelligence''
        \item \textbf{based on neuroscience} of the brain
    \end{itemize}

    \vspace{0.5cm}

    \pause

    $\rightarrow$ Learning Algorithms \pause (of the brain)
\end{frame}


% \begin{frame}[c]{What is HTM? II}
%     \Large
%     Core themes:
%     \begin{itemize}[<+(1)->]
%         \item Learning
%         \item Inference
%         \item Prediction
%         \item The brain as prediction machine
%     \end{itemize}
% \end{frame}

\begin{frame}[c]{The brain as Prediction Machine}
    \Large
    \begin{itemize}[<+(1)->]
        \item Prediction of future sensory input
%         \item On every level of the hierarchy
        \item 'Anticipating' events
        \item multiple connected regions
%         \item Specific instances
        \item Invariant representations
        \item Hierarchies of Concepts
        \item A sense of location
    \end{itemize}
\end{frame}


\begin{frame}[c]{Attributes of HTM Algorithms}
    \begin{itemize}[<+(1)->]
        \item can store, learn, infer and recall higher-order sequences
        \item learns unsupervised time-based patterns in unlabeled data on continuous streams
        \item robust against noise
        \item can learn multiple patterns at once
        \item suited for prediction, anomaly detection, classification
        \item tested and implemented in software
        \item commercially used (anomaly detection, NLP)
    \end{itemize}
\end{frame}



