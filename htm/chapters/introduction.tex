
\begin{frame}[standout]
    Disclaimer: \pause I don't really know what I'm talking about.
\end{frame}

\begin{frame}[c]{Epistemic status}
    \Large
    \begin{itemize}[<+(1)->]
        \item Evolving theories
        \item Hypotheses partially verified
        \item Theories are constantly being updated
        \item I know too little to say this is the newest information
    \end{itemize}
\end{frame}

\section{Introduction}


\begin{frame}[c]{Why HTM?}
    
\end{frame}


\begin{frame}[c]{Definition of Intelligence}
    
\end{frame}


% Include Topics/Facts:
% 
% - explain SDRs
% - explain spatial & temporal pooler
% - explain HTM
% - explain reticular activating system
% - explain gradient for Autism and that it is more 'bottom up'
% - explain 'top-down' phenomena and common biases
% - explain theory: inhibited-booost: hallucination
% - The datastructure of the brain




% Roadmap (?)
%
% - First: What is Intelligence ? (Early path, why we want to know this and why this is the most likely path)
% - Introduce the biological neuron we will be talking about (multiple dendrites, one axon, ...)
% - Introduce (SDR) Sparse Distributed Representations, in-Depth
% - Introduce the CLA
% - Introduce HTMv3 !
% - Motor movements are just very strong predictions the body 'makes happen'
%   - Flow state is where everything is exactly as predicted
% - Mistakes that happen by accidential correlation: search for good example; mistaking something to be true
% - explain gradient for Einstein/Craziness, proving vs disproving in the brain -> THC is bad
% - Mistakes that happen by top-down enforcing (e.g. the dalmatine picture, scrambled words)
% - Location-based composition: This explains mind palaces!
% - Everything is a 'conceptspace' - 

% - Explain Priming based on predictive stuff (the priming-bias)
% - Explain biases based on these models
% - Explain the fact that smarter people have less Brain activity



% Invariance:
% a dog is a dog no matter how it looks

% Autoassociative:
% having little clues but seeing the whole thing

% hierarchical:
% ...


